\chapter{Innovative algorithms to use for the two projects}
\label{cha:algorithm}

\section{Project 1: 2D path finding}
\label{sec:algorithm_project1}

\subsection{Finding process}
Before the finding process could begin, the neuronal space should be generated. 
All obstacles points are set to 0 and will not be updated. 
All target points are set to 1 and will not be updated. 
Other points which represent feasible configurations will be set to 0 initially and updated in the finding process. 
The finding process is in fact the process of updating values of neurons step by step. 
Assuming N neurons exist in the neuronal space. 
Their value could be changed due to inputs from adjacent neurons and sensory input (like obstacles that appear suddenly, but this is not considered in this project). 
That is to say
\[ \sigma_i(t+1)=g(\sum_{j}^{N}{T_{ij}\sigma_j}) \]
Where g is a sigmoid function, T is a symmetric matrix storing the diffusion factors (discussed in detail in next chapter). 
The detailed study of g will not be presented in this report, we choose \(g(x)=\tanh(\frac{x}{3^n}) \)where n is the dimension of neuronal space. 
(\(3^n\) is the number of neurons that got summed).
Step by step, the value of all neurons are updated (except for obstacles and destination), when the origin neuron (the neuron which represents the origin point of the robot) has a value greater than zero (in Python we could chose 1e-15 as zero), the path is found. From the origin neuron we go, step by step, to the next adjacent neuron whose value is the biggest among all adjacent neurons. This will give a set of neurons from the origin neuron to the destination neuron which represent a feasible path. It is what we want from this method. 

\subsection{Modification of diffusion factor for energy saving}
The symmetric matrix T mentioned above is what we called diffusion factor here. 
It controls the influence of one neuron to another. 
Thus the difficulty of going from one neuron to another. 
Which could be interpreted as the time or energy consumption in our case. 
Since T is symmetric, the difficulty level of going form neuron 1 to neuron 2 equals that of going from neuron 2 to neuron 1. 
This is not always correct like climbing and descending. 
But this problem is not yet considered in this project. 

Due to lack of real robots and environments, we could not simulate diffusion factors with real interpretations. 
But it is possible to simulate its capacity. 


\section{Project 2: Multi-arm robot}
\label{sec:algorithm_project2}
\section{Multi-arm robot}
\label{sec:algorithm_project2}
In this project, comparing to the first one, some additional steps must be taken.
One is to calculate the robot’s “presence” which means the space occupied in the work space by the robot’s arms in each configuration.
It is necessary for calculating the neuronal space.
Obstacles includes, in addition to all points of the obstacles in XY dimension,
all configurations that the robot’s presence has overlapped space with any obstacles.

These calculations consume lots of computational resources.
So various approaches that can reduce this consumption by thousands of times are proposed in the following chapters.

\subsection{Robot presence calculation}
\label{ssec:algorithm_presence}
The robot arms’ special presence is calculated using two methods.
One is to consider the distance between any point to the arm’s joint and to the arm’s central axis.
The other is to use the equation of robots outer lines.

It is tested that the first one is faster and the result of second one is better.
As we use a pre-calculation approach (introduced in section \ref{sec:optimizaiton_Precalculation}) to reduce the calculation time
in this step there is no need to consider the computational resources consumed in this part.
So we choose the second method.

Based on the joint coordinates, joint angles and arm size, the equations of the robot's arm's outer lines can be obtained.

\subsection{Preparation of neuronal space}
\label{ssec:algorithm_project2_space}
This preparation consists of calculating all the feasible configurations of the robot using the presences calculated.
All neurons of configuration that has overlap with any obstacles are fixed to 0.

This takes fairly amount of time.
So many computational approaches that accelerates the calculation are applied.
The most powerful ones are the usage of Numba and the vectorization. (see chapter \ref{cha:optimizaiton})
