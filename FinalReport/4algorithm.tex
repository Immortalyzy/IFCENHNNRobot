\chapter{Innovative algorithms to use for the two projects}
\label{cha:algorithm}

\section{Energy-saving path finding}
\label{sec:algorithm_project1}

\section{Multi-arm robot}
\label{sec:algorithm_project2}
In this project, comparing to the first one, some additional steps must be taken.
One is to calculate the robot’s “presence” which means the space occupied in the work space by the robot’s arms in each configuration.
It is necessary for calculating the neuronal space.
Obstacles includes, in addition to all points of the obstacles in XY dimension,
all configurations that the robot’s presence has overlapped space with any obstacles.

These calculations consume lots of computational resources.
So various approaches that can reduce this consumption by thousands of times are proposed in the following chapters.

\subsection{Robot presence calculation}
\label{ssec:algorithm_presence}
The robot arms’ special presence is calculated using two methods.
One is to consider the distance between any point to the arm’s joint and to the arm’s central axis.
The other is to use the equation of robots outer lines.

It is tested that the first one is faster and the result of second one is better.
As we use a pre-calculation approach (introduced in section \ref{sec:optimizaiton_Precalculation}) to reduce the calculation time
in this step there is no need to consider the computational resources consumed in this part.
So we choose the second method.

Based on the joint coordinates, joint angles and arm size, the equations of the robot's arm's outer lines can be obtained.

\subsection{Preparation of neuronal space}
\label{ssec:algorithm_project2_space}
This preparation consists of calculating all the feasible configurations of the robot using the presences calculated.
All neurons of configuration that has overlap with any obstacles are fixed to 0.

This takes fairly amount of time.
So many computational approaches that accelerates the calculation are applied.
The most powerful ones are the usage of Numba and the vectorization. (see chapter \ref{cha:optimizaiton})
