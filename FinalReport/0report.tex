%% * Format settings ****************************************************************
%% paper arrangement and main font size
\documentclass[utf8,twoside, 12pt, titlepage, openright]{report}
%% Paper size and margins
\usepackage[a4paper,top=2cm, bottom=3cm, left=3cm, right=2cm]{geometry}
%% Police
\usepackage{fontspec}
\setmainfont{Times New Roman}
%% Title format
\usepackage{titlesec}
\titleformat{\chapter}[block]
  {\normalfont\fontsize{16}{16}\bfseries\filcenter}{}{1em}{}
\titleformat{\section}
  {\normalfont\fontsize{14}{14}\bfseries}{\thesection}{1em}{}
\titleformat{\subsection}
  {\normalfont\fontsize{12}{12}}{\thesubsection}{1em}{}
\titleformat{\paragraph}
  {\normalfont\fontsize{12}{12}}{}{1em}{}
%% Inter-line
\usepackage{setspace}
\onehalfspacing

%% Indent first paragraph
\usepackage{indentfirst}

%% for blank pages at end
\newcommand{\blankpage}{
\newpage
\thispagestyle{empty}
\mbox{}
\newpage
}

%% Add Chinese support
\usepackage{xeCJK}
\usepackage{ifplatform}
%% For macOS users, the fonts need additional configurations.
\ifmacosx
\setCJKmainfont[BoldFont=STHeiti,ItalicFont=STKaiti]{STSong}
\setCJKsansfont[BoldFont=STHeiti]{STXihei}
\setCJKmonofont{STFangsong}
\fi

% For logbook
\usepackage{tabularx}


%% * Format settings ****************************************************************
%% The amssymb package provides various useful mathematical symbols
\usepackage{amssymb}
%% The amsthm package provides extended theorem environments
\usepackage{amsthm}
\usepackage{amsrefs}
\usepackage{mathrsfs}
\usepackage{amsmath}
\usepackage{bm}
\usepackage{subcaption}
%\usepackage{natbib}
\usepackage{graphicx}

%% The lineno packages adds line numbers. Start line numbering with
%% \begin{linenumbers}, end it with \end{linenumbers}. Or switch it on
%% for the whole article with \linenumbers after \end{frontmatter}.
%\usepackage{lineno}
%\bibliographystyle{model1-num-names}
%\bibliography{sample.bib}

\usepackage[hidelinks]{hyperref}

\begin{document}
\title{Application of Hopfield neural network on the energy-saving robot path finding and robot arm control.}
\author{Jérémie YANG Zhenyu 杨振宇 \\ Student number : 19214064}
\maketitle
\cleardoublepage
\setcounter{page}{1}

\tableofcontents
\newpage
\chapter*{Abstract}
\addcontentsline{toc}{chapter}{Abstract}
\label{sec:abstractEng}

Robots are important tools in nuclear power plants since they could receive radiation without doing
damage other than economy.
It could be foreseen that once a material for robot fabricating that is radiation-resist enough,
robot will be very universal in nuclear power plants.
On the other hand, algorithms controlling the robot will be very important too.
The complicated ground conditions in nuclear power plants require the robot to be able to climb stairs,
avoid or step over various kinds of obstacles and make right decisions for saving energy.


\chapter{Introduction}
\label{cha:introduction}
\section{Literature review}
\label{sec:review}

General intelligent robot controlling systems for legged robots contain path planning and gait planning parts. 

The path planning parts are similar with wheeled robots that could avoid obstacles and choose the optimum path from one place to another.
This part has been excessively studied, with various tools. 
The navigation algorithms of robots are divided into three categories\cite{RN9} :
deterministic algorithms, nondeterministic algorithms and evolutionary algorithms (usually a combination of two previous algorithms). 

The gait planning parts are only for legged robots. 
Human does the gait planning all the time without knowing it because our brain has learning to plan gaits in different conditions when we were infants. 
It all became automatic after the body has remembered it. 
However, it's not so easy to make robots do the same thing.

% Gap scope -- meaning of the work, the small circle in which we perform our work, important but not studied.
Hopfield neural network is not traditionally used for path finding or gait finding problems, but it turns out to be able to handle both. 
Such a model was proposed in 1995 \cite{RN11} that is very similar to a Hopfield neural network. 
It has topologically organized with nonlinear analog neurons. 
Its functions are not fully exploited, including the possibility of including energy consumption in path finding and gait planning while the robot could move with wheels. 
In addition, it could be applied in changing environment, the neuronal space can be updated while running.
% Add Methods, 
\section{Methods}
\label{sec:method}
Traditional intelligent path finding algorithm using deep learning is an extremely hot topic and very well studied.
In addition, they usually require a complete set of robot researching tools including a real robot which we do not possess.
So we didn't choose to work with deep neural network but the Hopfield neuronal network.
The principles and theory of Hopfield network are introduced in chapter \ref{cha:hopfield}.
It is simple to setup and we've anticipated its potential additional functionalities.
One is to use Hopfield's \textit{diffusion factor} to take account of energy or time consumption.
The other is to couple robot arm movement and space translation to globally control wheeled robots with arms.
The two possible potential functionalities will be introduced in following two projects.


\section{Projects to implement}
\label{sec:projects}

\subsection{Project 1 - 2D path finding problem}
\label{ssec:project1}

Project 1 consists of using Hopfield neuronal space to find an optimum path from an origin to a destination on a 2D map
on considering energy and time consumption.
Obstacles will be divided into 3 groups according the difficulty to pass them (in the sense of energy and time consumption).
There are obstacles easy to pass, difficult to pass and impossible to pass.
The robot will try to detour the obstacles but if the detour is slightly long, it will choose to pass the easy obstacles.
If the detour is too long, it will choose to pass the difficult obstacles.
Detailed description of the algorithm and results could be found at section \ref{sec:algorithm_project1} and
\ref{sec:result_project1}.


\subsection{Project 2 - Multi-arm target reaching problem}
\label{ssec:project2}

Project 2 consists of using Hopfield neuronal space to find an optimum configuration sequence for a robot
with two-joint arm and XY dimension translation to reach a target in a 2D space.
There will be obstacles that the robot should not touch and every point of the robot could reach the target
(This is configurable).
The robot could move freely on the plane and rotate its arms to which ever angle.
This could help controlling robots that move with wheels and have arms to perform certain tasks.
Detailed description of the algorithm and results could be found at section \ref{sec:algorithm_project2} and
\ref{sec:result_project2}.

\chapter{Hopfield neutral space}
\label{cha:hopfield}

\section{Principles}
\label{sec:hopfield_Principles}
Firstly, let us introduce the basic concepts of Hopfield neuronal space that are used in this project.
Work space: The real space where the robot is working. The dimension of this space is usually two or three.
Configuration space: A virtual finite space where every point is a configuration of the robot. Like the position or arm joint angle. The dimension of this space equals the freedom degree of the robot. (Note that the space can be continue or discrete, finite does not mean that there are countable number of points)
Obstacles: In the work space and the configuration space, they are places that the robot cannot reach.
Neuronal space: It is a discrete topologically ordered representation of the configuration space. Each point (which will be called “neuron”) represents a configuration of the robot and the robot can go from every point to its adjacent point directly. Thus a path in the neuronal space will represent a feasible path for the robot.
Neuron: Points of the neuronal space, each neuron is given a value between 0 and 1. The value will update in the finding process and finally indicate a path.

\section{Advantages and disadvantages}
\label{sec:hopfield_advanddisadv}
Comparing with deep learning and neural network that are in plain development at present,
Hopfield neuronal space has a very distinctive specialty – it does not require learning.
Due to this, the run time performance is usually not as good as other neural networks.



\chapter{Innovative algorithms to use for the two projects}
\label{cha:algorithm}

\section{Project 1: 2D path finding}
\label{sec:algorithm_project1}

\subsection{Finding process}
Before the finding process could begin, the neuronal space should be generated. 
All obstacles points are set to 0 and will not be updated. 
All target points are set to 1 and will not be updated. 
Other points which represent feasible configurations will be set to 0 initially and updated in the finding process. 
The finding process is in fact the process of updating values of neurons step by step. 
Assuming N neurons exist in the neuronal space. 
Their value could be changed due to inputs from adjacent neurons and sensory input (like obstacles that appear suddenly, but this is not considered in this project). 
That is to say
\[ \sigma_i(t+1)=g(\sum_{j}^{N}{T_{ij}\sigma_j}) \]
Where g is a sigmoid function, T is a symmetric matrix storing the diffusion factors (discussed in detail in next chapter). 
The detailed study of g will not be presented in this report, we choose \(g(x)=\tanh(\frac{x}{3^n}) \)where n is the dimension of neuronal space. 
(\(3^n\) is the number of neurons that got summed).
Step by step, the value of all neurons are updated (except for obstacles and destination), when the origin neuron (the neuron which represents the origin point of the robot) has a value greater than zero (in Python we could chose 1e-15 as zero), the path is found. From the origin neuron we go, step by step, to the next adjacent neuron whose value is the biggest among all adjacent neurons. This will give a set of neurons from the origin neuron to the destination neuron which represent a feasible path. It is what we want from this method. 

\subsection{Modification of diffusion factor for energy saving}
The symmetric matrix T mentioned above is what we called diffusion factor here. 
It controls the influence of one neuron to another. 
Thus the difficulty of going from one neuron to another. 
Which could be interpreted as the time or energy consumption in our case. 
Since T is symmetric, the difficulty level of going form neuron 1 to neuron 2 equals that of going from neuron 2 to neuron 1. 
This is not always correct like climbing and descending. 
But this problem is not yet considered in this project. 

Due to lack of real robots and environments, we could not simulate diffusion factors with real interpretations. 
But it is possible to simulate its capacity. 


\section{Project 2: Multi-arm robot}
\label{sec:algorithm_project2}
\chapter{Optimizations of computer code}
\label{cha:optimizaiton}

\section{Vectorization}
\label{sec:optimizaiton_vectorization}

\section{Numba and code modifications}
\label{sec:optimizaiton_numba}
\chapter{Results and discussion}
\label{cha:result}

\section{Energy-saving path finding}
\label{sec:result_project1}
The test environment is a 100x100 2D space given in figure \ref{fig:envs}, 
Without any floating objects in the space, each point is dedicated a value indicating the height of the obstacle of the point (0 means plain ground with no obstacles). 
For simplicity, only square obstacles are considered as shown in the figures.



\begin{figure}[!htb]
    \centering
    \begin{subfigure}[b]{.475\textwidth}
        \centering
        \includegraphics[width=\textwidth]{Figs/env1.png}
        \caption{The robot should pass the green obstacle}
        \label{fig:env1}
    \end{subfigure}%
    \hfill
    \begin{subfigure}[b]{.475\textwidth}
        \centering
        \includegraphics[width=\textwidth]{Figs/env2.png}
        \caption{The robot should avoid the green obstacle}
        \label{fig:env2}
    \end{subfigure}
    \vskip\baselineskip
    \begin{subfigure}[b]{.475\textwidth}
        \centering
        \includegraphics[width=\textwidth]{Figs/env3.png}
        \caption{The robot should choose to detour}
        \label{fig:env3}
    \end{subfigure}%
    \hfill
    \begin{subfigure}[b]{.475\textwidth}
        \centering
        \includegraphics[width=\textwidth]{Figs/env4.png}
        \caption{The robot should pass the purple obstacle}
        \label{fig:env4}
    \end{subfigure}
    \caption{Test environments}
    \label{fig:envs}
\end{figure}

Green color represents obstacles that are easy to pass. 
So the robot should choose to pass the green obstacles if the detour is not too long. (figure \ref{fig:env1} and \ref{fig:env2})

Purple color represents obstacles that are difficult to pass and red color represents obstacles that are impossible to pass.
So the robot should choose to detour around purple obstacles unless the detour is extremely long. (figure \ref{fig:env3} and \ref{fig:env4})
The parameters should be chosen to make the robot make same decisions as indicated in the captions.

\section{Multi-arm robot}
\label{sec:result_project2}


\blankpage
\blankpage

\end{document}