\chapter{Introduction}
\label{cha:introduction}
\section{Literature review}
\label{sec:review}

General intelligent robot controlling systems for legged robots contain path planning and gait planning parts. 
The path planning parts are similar with wheeled robots 

% Gap scope -- meaning of the work, the small circle in which we perform our work, important but not studied.
% Add Methods, 
\section{Methods}
\label{sec:method}
Traditional intelligent path finding algorithm using deep learning is an extremely hot topic and very well studied.
In addition, they usually require a complete set of robot researching tools including a real robot which we do not possess.
So we didn't choose to work with deep neural network but the Hopfield neuronal network.
The principles and theory of Hopfield network are introduced in chapter \ref{cha:hopfield}.
It is simple to setup and we've anticipated its potential additional functionalities.
One is to use Hopfield's \textit{diffusion factor} to take account of energy or time consumption.
The other is to couple robot arm movement and space translation to globally control wheeled robots with arms.
The two possible potential functionalities will be introduced in following two projects.


\section{Projects to implement}
\label{sec:projects}

\subsection{Project 1 - 2D path finding problem}
\label{ssec:project1}

Project 1 consists of using Hopfield neuronal space to find an optimum path from an origin to a destination on a 2D map
on considering energy and time consumption.
Obstacles will be divided into 3 groups according the difficulty to pass them (in the sense of energy and time consumption).
There are obstacles easy to pass, difficult to pass and impossible to pass.
The robot will try to detour the obstacles but if the detour is slightly long, it will choose to pass the easy obstacles.
If the detour is too long, it will choose to pass the difficult obstacles.
Detailed description of the algorithm and results could be found at section \ref{sec:algorithm_project1} and
\ref{sec:result_project1}.


\subsection{Project 2 - Multi-arm target reaching problem}
\label{ssec:project2}

Project 2 consists of using Hopfield neuronal space to find an optimum configuration sequence for a robot
with two-joint arm and XY dimension translation to reach a target in a 2D space.
There will be obstacles that the robot should not touch and every point of the robot could reach the target
(This is configurable).
The robot could move freely on the plane and rotate its arms to which ever angle.
This could help controlling robots that move with wheels and have arms to perform certain tasks.
Detailed description of the algorithm and results could be found at section \ref{sec:algorithm_project2} and
\ref{sec:result_project2}.
