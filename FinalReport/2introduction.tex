\chapter{Introduction}
\label{cha:introduction}
\section{Literature review}
\label{sec:review}

\section{Choice of technique}
\label{sec:choice}

\section{Projects to implement}
\label{sec:projects}

\subsection{Project 1 - 2D path finding problem}
\label{ssec:project1}

Project 1 consists of using Hopfield neuronal space to find an optimum path from an origin to a destination on a 2D map
on considering energy and time consumption.
Obstacles will be divided into 3 groups according the difficulty to pass them (in the sense of energy and time consumption).
There are obstacles easy to pass, difficult to pass and impossible to pass.
The robot will try to detour the obstacles but if the detour is slightly long, it will choose to pass the easy obstacles.
If the detour is too long, it will choose to pass the difficult obstacles.


\subsection{Project 2 - Multi-arm target reaching problem}
\label{ssec:project2}

Project 2 consists of using Hopfield neuronal space to find an optimum configuration sequence for a robot
with two-joint arm and XY dimension translation to reach a target in a 2D space.
